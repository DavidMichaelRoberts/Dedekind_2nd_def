\documentclass[leqno,hidelinks,a4paper]{article}
\usepackage[german,english]{babel}
\usepackage{amsmath,amssymb,amsthm}
\usepackage{color,hyperref}
\usepackage{enumerate}
\usepackage{microtype}

\definecolor{darkblue}{rgb}{0,0,.7}
\definecolor{ultrav}{rgb}{0.5,0,1}

\hypersetup{colorlinks,
            breaklinks,
            linkcolor=ultrav,
            urlcolor=darkblue,
            anchorcolor=darkblue,
            citecolor=darkblue}

\newtheoremstyle{}
        {}%space above
        {}%space below
        {}%body font
        {}%Indent amount
        {\bfseries}%Theorem head font
        {}%Punctuation after theorem head
        {.5em}%Space after theorem head
        {#1.#2.~\thmnote{#3.}
}%Theorem head spec
\swapnumbers

\theoremstyle{definition}

\newtheorem{satz}{\protect\satzname}
\newtheorem{deff}[satz]{\protect\deffname}
\newtheorem*{zusatz}{\protect\zusatzname}
\newtheorem*{hilfssatz}{\protect\hilfssatzname}

\newcommand{\satzname}{}
\newcommand{\deffname}{}
\newcommand{\zusatzname}{}
\newcommand{\hilfssatzname}{}

\addto\captionsgerman{%
    \renewcommand{\satzname}{\hspace{-4pt}.\ Satz}%
    \renewcommand{\deffname}{\hspace{-4pt}.\ Definition}%
    \renewcommand{\zusatzname}{Zusatz}%
    \renewcommand{\hilfssatzname}{Hilfssatz}%
}

\addto\captionsenglish{%
    \renewcommand{\satzname}{\hspace{-4pt}.\ Theorem}%
    \renewcommand{\deffname}{\hspace{-4pt}.\ Definition}%
    \renewcommand{\zusatzname}{Corollary}%
    \renewcommand{\hilfssatzname}{Lemma}%
}

\newcommand\Beweis{\newline $ \phantom{'.'} \rhd \ $}%
\newcommand\beweis{ $ \phantom{'.'} \rhd \ $}%

% From https://narkive.com/P4sNM0k5:7.443.63
\newcommand\TeilVon{\mathrel{\raisebox{0.45ex}{$\mathfrak{3}$}}}
\newcommand\partof{\mathrel{\raisebox{0.45ex}{$\mathfrak{3}$}}}

\pagestyle{myheadings}
\markright{Dedekind, Second definition of the finite and infinite\hfill}

% for bilingual version
% \usepackage{paracol}
% \usepackage[landscape, total={8in, 8in}]{geometry}
% \usepackage{showframe}


\begin{document}
\title{Second definition (1899.3.9) of the finite and infinite}
\author{Richard Dedekind\thanks{First pass machine translation, tweaked by David Michael Roberts, March 2024. The start of page number $n$ in the original is denoted by \textbf{[n]} in the current text. Thanks to \ldots for improvement comments}}
\date{}
\maketitle

First published in the second edition (1893) of the text ``Was sind und was sollen die Zahlen?'' page XVII, in the form:

\begin{quote}
A system $S$ is called finite if it can be mapped into itself in such a way that no proper part of $S$ is mapped into itself; in the opposite case, $S$ is called an infinite system.
\end{quote}

Pursuing this definition of a finite system $S$ \emph{without using the natural numbers}.
Let $\phi$ be a mapping of $S$ into itself, through which no proper part [echter Teil] of $S$ is mapped into itself.
Small Latin letters $a, b \ldots z$ always mean elements of $S$, capital Latin letters $A, B \ldots Z$ mean \emph{parts} [Teile] of $S$; the images of \ldots $a, A$ generated by $\phi$ are respectively denoted by $a', A'$.
That $A$ is part of $B$ is denoted by $A \partof B$.\footnote{[\emph{In modern notation, $\subseteq$. Note also that Dedekind doesn't permit a `part' to be empty, so that strictly speaking $A\partof B$ means $\emptyset \neq A \subseteq B$. ---DMR}]} The system consisting of the elements $a, b, c \ldots $ is denoted by $[a, b, c \ldots]$.
So it [\emph{the definition of finiteness}] is
\begin{equation}\label{eq1}
				S' \partof S
\end{equation}
and
\begin{equation}\label{eq2}
		\text{from } A' \partof A \text{ it follows that } A = S.
\end{equation}

\begin{satz}\label{thm1}
$S' = S$. Every element of $S$ is an image of (at least) one element $r$ of $S$. Because from \eqref{eq1} it follows $(S')' \partof S'$, hence by \eqref{eq2}, our theorem holds.
\end{satz}

Every system $[s]$ consisting of a single element $s$ is finite because it has no proper part and is mapped into itself by the identity mapping.
This case is excluded below, $S$ means a finite system that does not consist of a single element.

\begin{satz}\label{thm2}
Every element $s$ is different from its image $s'$, in symbols: $s \neq s'$.
Because if $s = s'$, then $[s]' = [s'] = [s] \partof [s]$, so according to \eqref{eq2} also $[s] = S$ in contradiction to our assumption about $S$.
\end{satz}

\textbf{[451]}


\begin{deff}\label{def3}
If $s$ is a certain element of $S$, then $H_s$ shall be used to denote any part of $S$ that satisfies the following two conditions:
\begin{enumerate}[I.]
\item $s$ is element of $H_s$, so $[s] \partof H_s$, also
\[
	[s] + H_s = H_s.
\]
\item If $h$ is an element of $H_s$ different from $s$, then $h'$ is also an element of $H_s$; So if $H \partof H_s$, but $s$ is not contained in H, then $H' \partof H_s$.
\end{enumerate}
\end{deff}

 \begin{satz}\label{thm4}
 $S$ and $[s]$ are special systems $H_s$, and $[s]$ is the intersection [Durchschnitt] (the commonality Gemeinheit\footnote{[\emph{Beman's 1901 translation is `community' ---DMR}]}) of all systems $H_s$ corresponding to the element s. Obvious.\end{satz}

\begin{satz}\label{thm5}
$H_s = S$ or a proper part of $S$, depending on whether $s'$ lies in $H_s$ or not.--- For if $s'$ lies in $H_s$, then it follows from II in \ref{def3}. that $H'_s \partof H_s$, therefore by \eqref{eq2} that $H_s = S$; and vice versa, if $H_s = S$, then $s'$ also lies in $H_s$.\end{satz}



\begin{satz}\label{thm6}
 If $H_s$ is a proper part of $S$, then $s'$ is the only element of $H'_s$ that lies outside $H_s$.
--- Because every element $k$ of $H'_s$ is image $h'$ of at least one element $h$ in $H_s$;
If $k=h'$ is different from $s'$, then $h$ is also different from $s$, and consequently (according to II in \ref{def3}.) $k = h'$ lies in $H_s$, while the element $s'$ of $H'_s$ (according to \ref{thm5}.) lies outside $H_s$.
\end{satz}

\begin{satz}\label{thm7}
Every system $H'_s$ is a system $H_{s'}$, that is (Definition \ref{def3}.):
\begin{enumerate}[I'.]
	\item $s'$ is element of $H'_s$
	\item If $k$ is an element of $H'_s$ that is different from $s'$, then $k'$ also lies in $H'_s$.
\end{enumerate}\end{satz}

The first follows from the fact that $s$ lies in $H_s$, the second from the fact that $k$ lies in $H_s$ (Theorem (\ref{thm6})).

\begin{satz}\label{thm8}
If $A$, $B$, $C$ \ldots\ are special systems $H_s$ corresponding to the same $s$, then their intersection $H$ is also a system $H_s$.\end{satz}

Because according to \ref{def3}.I. $s$ is a common element of $A, B, C, \ldots$, and therefore also an element of $H$.
Furthermore, if $h$ is an element of $H$ that is different from $s$, then (according to \ref{def3}.II.) the image $h'$ is an element of $A$, of $B$, of $C$, \ldots, and therefore also of $H$.
$H$ therefore fulfills the two conditions I, II in \ref{def3}.\ that are characteristic of every $H_s$.

\begin{deff}\label{def9}
If $a$, $b$ are certain elements of $S$, then the symbol $ab$ (\emph{segment} [Strecke] $ab$) should mean the intersection [Durchschnitt] of all those systems $H_b$  which (such as $S$) contain the element $a$.
\end{deff}

\noindent \textbf{[452]}

\begin{satz}\label{thm10}
$a$ is an element of ab, i.e. $[a] \partof ab$. Because $ab$ is the intersection of all systems $H_b$ in which $a$ lies. - (a is the \emph{start} [Anfang] of ab.)
\end{satz}

\begin{satz}\label{thm11}
$ab$ is a system $H_b$, i.e. $[b] \partof ab$, and if $s$ is an element of $ab$ different from $b$, then $[s'] \partof ab$.
--- This follows from \ref{thm8}. --- So $b$ is an element (the \emph{end} [Ende]) of $ab$. If $H \partof ab$ but $b$ is not contained in $H$, then $H' \partof ab$.
\end{satz}

\begin{satz}\label{thm12}
 From $[a] \partof H_b$, follows from $ab \partof H_b$.
Immediate consequence of \ref{def9}.
\end{satz}

\begin{satz}\label{thm13}
$aa = [a]$.
This follows from \ref{thm4}., because $aa$ is the intersection of all $H_a$ that contain the element $a$ (according to \ref{def3}. I.).
\end{satz}

\begin{satz}\label{thm14}
If $b'$ is an element of $ab$, then $ab = S$.
--- This follows from \ref{thm11} and \ref{thm5}.
\end{satz}


\begin{satz}\label{thm15}
 $b' b = S$.
 --- This follows from \ref{thm14} and \ref{thm10}.
\end{satz}

\begin{satz}\label{thm16}
If $c$ is an element of $ab$, then $cb \partof ab$.
--- This follows from \ref{thm12}, because $ab$ is an $H_b$ (according to \ref{thm11}) which contains the element $c$.
\end{satz}

\begin{satz}\label{thm17}
If $A + B$ means the system composed of $A, B$, then
\[
	a'b + b'a = S.
\]
For if $s$ is an element of $a'b$, then $s'$ is contained in $b'a$ or $a'b$, depending on $s = b$ or different from $b$ (according to \ref{thm10} or \ref{thm11} and \ref{def3}. II), and likewise if $s$ is an element of $b'a$, then $s'$ is contained in $a'b$ or $b'a$; therefore $(a'b + b'a)' \partof a'b + b'a$; This leads to the theorem according to \eqref{eq2}.
\end{satz}

\begin{satz}\label{thm18}
If $a$ is different from $b$, then $ab = [a] + a'b$.
For since $a$ is an element of $ab$ different from $b$, then $a'$ is an element of $ab$ (by \ref{thm10}, \ref{thm11}), and consequently (by \ref{thm16}) $a'b \partof ab$; since furthermore (by \ref{thm10}) we also have $[a] \partof ab$, therefore
\[
	[a] + a'b \partof ab.
\]
Furthermore: every element $s$ of $[a] + a'b$ that is different from $b$ is either $= a$ or an element of $a'b$ that is different from $b$, in both cases $s'$ is (by \ref{thm10}, \ref{thm11}) an element of $a'b$, therefore also from $[a]+ a'b$, and since (by \ref{thm11}) also $[b] \partof [a] + a'b$, then $[a] + a'b$ is a system $H_b$; Finally, since $[a] \partof [a] + a'b$, so (by \ref{thm12})
\[
	ab \partof [a] + a' b.
\]
The theorem follows from the comparison of both results.
\end{satz}


\noindent \textbf{[453]}

\begin{satz}\label{thm19}
If $a, b$ are different elements of $S$, then $a$ lies outside $a'b$, and $b$ lies outside $b'a$.\end{satz}

\begin{proof}
If one assumes the opposite, that there is an element $a$ that is different from $b$ and lies in $a'b$, and that $A$ denotes the system of all such elements $a$, the following holds.
If one puts $a'=s$, then $a$ lies in $sb$, and since $a$ is different from $b$, and therefore (according to \ref{thm13}) is not in $bb$, then $s$ is different from $b$, and from this it follows (according to \ref{thm18}) that $sb = [s] + s 'b$.
Furthermore, since $a$ (according to \ref{thm2}) is different from $s$ and lies in $sb$, then $a$ must lie in $s'b$, and from this it follows again (according to \ref{thm1}) that $s$ (as the image $a'$) also lies in $s'b$.
Therefore, the image $a'$ of every element $a$ of $A$ is also contained in $A$, i.e. $A' \partof A$.
But since $A=S$ would follow from this, while $A$ does not contain the element $b$, our assumption is inadmissible, so the theorem is true, Qed.
The second part follows by exchanging $a$ with $b$.\end{proof}

\begin{satz}\label{thm20}
If $a$, $b$ are different, then the segments $a'b$, $b'a$ have no common element.
\end{satz}

\begin{proof}
If one assumes the opposite, that there is a common element $m$ of $a'b$, $b'a$, then it follows from the preceding Theorem \ref{thm19} that $m$ is different from $b$ and from $a$; therefore (according to \ref{thm11}) the image $m'$ must also be a common element of $a'b$ and $b'a$;
Therefore, if M denotes the system of all such elements $m$, then $M' \partof M$, i.e. $M=S$. But this is impossible because $a$, $b$ are elements of $S$ but not elements of $M$. So our theorem is true.
\end{proof}

\begin{satz}\label{thm21}
If $a$, $b$ are different, then the images $a'$, $b'$ are also different.
\end{satz}

\begin{proof}
Otherwise the segments $a'b$, $b'a$ would have a common element $a'=b'$, because (according to \ref{thm10}) $a'$ is an element of $a'b$ and $b'$ is an element of $b'a$.
\end{proof}

\begin{satz}\label{thm22}
From $cb=S$ follows $c = b$.
\end{satz}

\begin{proof}
There is (according to \ref{thm1} and \ref{thm21}) in $S$ one and only one element $a$ which satisfies the condition $a'=c$, and therefore $a'b = S$, therefore $[a] \partof a'b$; Therefore (by \ref{thm19}) $a=b$, thus $c=b'$, Qed.
\end{proof}

\begin{satz}\label{thm23}
If $a$, $b$ are different, then every element of $S$ is contained in one and only one of the segments $a'b$, $b'a$. --- This follows from \ref{thm17} and \ref{thm20}.
\end{satz}

\noindent \textbf{[454]}

\begin{satz}\label{thm24}
If $a$, $b$, $c$ are different, then the segments $b'c$, $c'a$, $a'b$ have no common element, and the same applies to the segments $a'c$, $b'a$, $c'b$.
\end{satz}
\begin{proof}
Because the opposite assumption, that there is an element $m$ common to the segments $b'c$, $c'a$, $a'b$, leads to a contradiction.
Let $M$ be the system of all such elements.
Since (according to \ref{thm19}) $a$ is not in $a'b$, $b$ is not in $b'c$, $c$ is not in $c'a$, then $m$ is different from $c$, $a$, $b$, and consequently (by \ref{thm11}) $m'$ is a common element of $b'c$, $c'a$, $a'b$, i.e. an element of $M$; therefore $M' \partof M$, hence  $M=S$.
But this is impossible because $M$ does not contain any of the elements $a$, $b$, $c$. So our theorem is true.
--- The second part results from the first if one swaps $a$ with $b$, which does not change the assumption.\end{proof}

\begin{zusatz}
If you put (as in the following \ref{thm25}):
\[
    A=c'b,\ B=a'c,\ C = b'a;\ A_1 = b'c,\ B_1 = c'a,\ C_1 = a'b,
\]
then $A-B-C=0$\footnote{[The symbol means the intersection [Durchschnitt].]} (empty) and $A_1- B_1 - C_1=0$ (empty) and (according to \ref{thm17}, \ref{thm20}) hence
\[
S = A + A = B + B_1 = C +C_1;
\]
\[
0=A-A_1=B-B_1=C-C_1.
\]
This also applies (according to \ref{thm20}) if at least two of the elements a, b, c are different.\end{zusatz}

\begin{satz}\label{thm25}
If $a$, $b$, $c$ are different, then one and only one of the following two cases occurs: Either
\begin{multline}
b'c=b'a+a'c,\quad c'a =c'b+b'a,\quad a'b = a'c+c'b \\
c'b=c'a-a'b,\quad a'c=a'b-b'c,\quad b'a=b'c-c'a
\end{multline}
and each element of S lies in one, but only one, of the segments c'b, a'c, b'a; or
\begin{multline}
c'b=c'a+a'b,\quad a'c=a'b+b'c,\quad b'a=b'c + c'a \\
b'c=b'a-a'c,\quad c'a=c'b-b'a,\quad a'b=a'c-cb
\end{multline}
and each element of $S$ lies in one, but only one, of the segments $b'c$, $c'a$, $a'b$.\end{satz}

\begin{proof}
According to \ref{thm23}, $c$ lies either in $a'b$ or in $b'a$. We only consider the first case because the second arises from it by exchanging $a$ for $b$.
Since $c$ is in $a'b$ and is distinct from $b$, \textbf{[455]} then (according to \ref{thm11}) $c'$ also lies in $a'b$, and consequently (by \ref{thm16}) $c'b \partof a'b$; from this it follows (by \ref{thm19}) that $c'b$ has no element in common with $b'a$; now (by \ref{thm17}) is $a'b+b'a=b'c+c'b$, therefore $b'a \partof b'c$, and consequently (by \ref{thm11}) $a$ is in $b'c$.
From the assumption that $c$ lies in $a'b$, it follows: $c'b \partof a'b$, $b'a \partof b'c$, $a$ lies in $b'c$.
In the same way, this last conclusion follows if one assumes $c$, $a$, $b$ replaced by $a$, $b$, $c$, respectively, again we have the consequences $a'c \partof bc$, $cb \partof c'a$, and that $b$ lies in $c'a$; and from this it follows again $b'a \partof c'a$, $a'c \partof a'b$ (and the first assumption: $c$ lies in $a'b$).

\bigskip

[Circular diagrams]

\bigskip

Therefore: $c'b \partof a'b$, $b'a \partof b'c$, $a'c \partof b'c$, $c'b \partof ca$, $b'a \partof c'a$, $a'c \partof a'b$, and also $b'a + a'c \partof b'c$, $c'b + b'a \partof c'a$, $a'c + c'b \partof a'b$.
If now an element of, say, $b'c$ is neither in $b'a$ nor in $a'c$, then (according to \ref{thm23}) it would be a common element of $b'c$, $a'b$, $c'a$, which (according to \ref{thm24}) is impossible; therefore $b'c \partof b'a + a'c$, therefore also $b'c=b'a + a'c$, and likewise follows $c'a=c'b+b'a$, $a'b=a'c+c'b$. If, say, $b'a$, $a'c$ have a common element, then the same would also be a common element of $b'c$, $c'a$, $a'b$, which (according to \ref{thm24}) is not the case.
From $S=b'c+c'b$ we finally get $S=b'a+a'c+c'b$, which means our theorem is completely proven.
\end{proof}

\begin{zusatz}
It can never be that $[a] \partof cb$ and $[b] \partof ca$ at the same time; because (according to \ref{thm18}) then $[a] \partof c'b$ and $[b] \partof c'a$ would have to be at the same time, which is impossible.
\end{zusatz}
\begin{satz}\label{thm26}
From $ab=cb$ it follows that $a=c$, and if $ab=cd$ is a proper part of $S$, then $a=c$, $b = d$.
\end{satz}

This follows from earlier theorems.
Since (by \ref{thm10}) $c$ is in $cb$, and therefore also in $ab$, then if $a=b$, then $ab = [a]$, then must also $c=a$.
But if $a$ is different from $b$, then (by \ref{thm18}) $ab = [a]+a'b$, \textbf{[456]} and (by \ref{thm19}) $a'b$ is a proper part of $ab$;
If one assumes that $c$ is different from $a$, then $c$ must lie in $a'b$, so (by \ref{thm16}) $cb \partof a'b$, i.e. $cb$ is a proper part of $ab$;
But since $cb=ab$, this assumption is inadmissible, and therefore always $c=a$, Qed.
Furthermore, if $ab=cd$ is a proper part of $S$, then $b=d$; If $b$ is different from $d$, then (by \ref{thm11}) $b'$ must also be in $cd$, and therefore also in $ab$; But then (by \ref{thm14}) would $ab = S$ violating the assumption, so $b=d$, therefore $ab=cb$, therefore also $a=c$, Qed.

\begin{satz}\label{thm27}
Every system $H$ (defined in \ref{def3}.) is a segment [Strecke] $a's$ with the end $s$ and its beginning $a'$ completely determined.
\end{satz}
\begin{proof}
If $H_s= S$, then $H_s= s's$ (according to \ref{thm15}).
But if $H_s$ is a proper part of $S$, then $A$ is the system of all elements of $S$ that lie outside $H_s$, so $S=A+H_s$.
Since $A$ is a proper part of $S$, then $A' \partof A$ cannot be, so there is certainly an element a in $A$, whose image $a'$ lies outside $A$, hence in $H_s$;
Since (according to \ref{thm12}) $a's \partof H_s$, then $a's$ and $A$ have no common element.
Since $a$ is in $A$, $s$ is in $H_s$ (even in $a's$), then $a$ and $s$ are different, so (by \ref{thm20}) the segments $a's$, $s'a$ have no common element, and (by \ref{thm17}) $a's+  s'a=S=H_s+ A,$ therefore $A \partof s'a$.
If one now assumes that $a's$ is a proper part of $H_s$, and $H$ denotes the system of all those elements of $H_s$ which are outside $a's$, i.e. in $s'a$, then $H_s = H+a's$, and $s'a = H + A$, so $H=H_s-s'a$ is the intersection of the systems $H_s$, $s'a$.
Since neither $s$ nor $a$ lies in $H$, it follows from (that?) $H \partof H_s$, and $H \partof s'a$ (according to \ref{def3}.\ and \ref{thm11}.) that $H' \partof H_s$ also, and $H' \partof s'a$, also $H' \partof H$, hence $H=S$.
But this is impossible because $s$ (and also $a$) lies outside $H$. Therefore certainly $H_s= a's$, and $A=s'a$, Qed.
\end{proof}

\begin{satz}\label{thm28}
The intersection of such segments $as, bs\ldots$ which have the same end $s$ is itself such a segment $hs$, and its beginning $h$ is completely determined.
\end{satz}

For every such segment is (according to \ref{thm11}) a system $H_s$, and (according to \ref{thm8}) the same applies to its intersection, from which the theorem (according to \ref{thm27}) follows.

\begin{zusatz}[\textbf{to 28}]
\label{cor_to_thm28}
% Corollary to 28.
The intersection of the segments $as, bs, cs \ldots$ is itself one of these segments. --- As a proof, let us first state the
\end{zusatz}

\begin{hilfssatz}
If $hs$ is a proper part of $as$, and $k$ is the element whose image is $k'=h$, then $hs$ is also a proper part of $ks$, and at the same time $ks \partof as$.
\end{hilfssatz}
\noindent \textbf{[457]}

\begin{proof}
If $k=s$, then $hs=s's= S$, while $hs$ is a proper part of $as$, and therefore also of $S$. Since $k$ is different from $s$, then (by 18) $ks =[k]+ hs$, and (according to 19) $k$ is not contained in $hs$, so $hs$ is a proper part of $ks$.
Since $hs$ is a proper part of as, let $as = M + hs$, where $M$ is the system of all elements $m$ of as that lie outside $hs$ and are therefore also different from $s$; from this follows $M' \partof as$, and since $M'$ obviously cannot be part of $M$ (because $M$ is not $= S$), there must be in $M$ an element $m$, the image $m'$ of which lies outside $M$, hence in hs, from which $m's \partof hs$ follows\footnote{[The proof is apparently incomplete. According to J. Cavaillès, a proof of the Lemma results directly from 25 by replacing the a, b, c there by a, k, s. The Corollary follows from 28 and the Lemma. E.N.]}.\end{proof}

\begin{satz}\label{thm29}
If $T$ is a part of $S$, and $s$ is an element of $S$, then in $S$ there is always one and only one associated element $s_1$, which has the following two properties:
\begin{enumerate}[1.]
	\item If $a$ satisfies the condition $T \partof as$, then $s_1s \partof as$
	\item $T \partof s_1s$
\end{enumerate}
and from this follow the two properties
\begin{enumerate}[1.]
	\setcounter{enumi}{2}
	\item $s_1$ is in $T$
	\item The segment $ss_1$ contains no element of $T$ that is different from $s$ and $s_1$.
\end{enumerate}
\end{satz}
\begin{proof}
Since $s's = S$, therefore $T \partof s's$ (by \ref{thm15}), there is at least one element $a$ that satisfies the condition $T \partof as$. If $A$ is the system of all such elements $a$, then (according to 28) the intersection of all the segments corresponding to them is a segment $s_1s$, where $s_1$ is a completely determined element of $S$.
According to the concept of an intersection, $s_1$ has the property 1., but also property 2., because $T$ is a common part of all as, and therefore also part of their intersection $s_1s$. If $s_1=s$, then $s_1s=ss=[s]$, then it follows from 2. that $T$ consists of the single element $s$; and vice versa, if $s$ lies in $T$ and is the only element of $T$, then $T = [s] = ss$, so then according to 1. $s_1s \partof ss$, therefore $s_1 = s$; In this case, $s_1$ therefore has the property 3.\ and obviously also the property 4.\ But if $s_1$ is different from $s$, then (according to \ref{thm18}) $s_1s= [s_1]+(s_1)'s$.
If one now assumes that $s_1$ lies outside $T$, if every element of $T$ is different from $s_1$, it follows from 2. also that $T \partof (s_1)'s$, and from this according to 1. we also have $s_1s \partof (s_1)'s$, but this is impossible because (according to \ref{thm10}) in $s_1s$ is the element $s_1$ (according to \ref{thm19}) which lies outside $(s_1)'s$; therefore
\textbf{[458]}
our assumption is inadmissible, i.e. $s_1$ has property 3.
We now consider the segment $ss_1$; if it has an element $u$ that is different from $s$ and $s_1$, then $s$ is also different from $s_1$ (because otherwise $ss_1 = [s]$, which would also give $u=s$), and (according to \ref{thm18}) $ss_1 = [s] + ss_1$; therefore $u$ lies in $s's_1$, therefore (according to \ref{thm19}) outside $(s_1)'s$, and since (as above) $s_1s= [s_1] + (s_1)'s$, and $u$ is also different from $s_1$, then $u$ also lies outside $s_1s$, therefore according to 2.\ also outside $T'$, i.e. $s_1$ also has property 4.
\end{proof}

\noindent \textbf{30.}\label{thm30} \hspace{-4pt} \emph{Mapping of $S$ into $T$}. Through \ref{thm29} a mapping $\psi$ of $S$ into $T$ is created, which is defined by the fact that each element $s$ of $S$ is sent by $\psi$ into the element $s_1$, which is defined there and (according to \ref{def3}.) lies in $T$.
If $A$ is then any part of $S$, then $A_1$ should mean the associated image of $A$ (i.e. the system of images $a_1$ of all elements $a$ of $A$). So $S_1 \partof T$, also $T_1 \partof T$, i.e. $T$ is mapped by $\psi$ into itself.

\stepcounter{satz}
\begin{satz}\label{thm31}
This mapping of $T$ into itself is a similar one, i.e.: if $a, b$ are different elements of $T$, then their images $a_1, b_1$ are also different.
\end{satz}

\begin{proof}
By \ref{thm29}, $T \partof a_1a$ and $T \partof b_1b$. Since $a, b$ are elements of $T$, then $[a] \partof b_1b$, $[b] \partof a_1a$. If, although $a, b$ are different, $a_1 = b_1  \dot{=}
%[this might be  scanning artefact, or maybe it's an inequality? Or a definition]
c$, so then $[a] \partof cb, [b] \partof c$; but since $c$ is different from $a$ and $b$ (because otherwise $a=b$), this is impossible (after the Corollary to \ref{thm25}). Therefore $a_1, b_1$ are different, Qed.
\end{proof}

\section*{Explanations to the above treatise}

The definition of the finite given here is chronologically the first that enables the derivation of all properties without using the axiom of choice --- a fact that Dedekind was probably not yet aware of.
He himself only draws the first conclusions; In this way, one can still conclude from his last theorem that every subset of a finite set is finite, and the principle of complete induction can be proven and thus move on to Dedekind's original definition (cf. a forthcoming note\footnote{[\emph{This paper appeared as ``Sur la deuxi\`eme des ensembles finis donn\'ee par Dedekind'', Fundamenta Mathematicae \textbf{19} (1932) pp 143--148. Available from \url{https://www.impan.pl/en/publishing-house/journals-and-series/fundamenta-mathematicae/all/19/0/92980/sur-la-deuxieme-des-ensembles-finis-donnee-par-dedekind} ---DMR}]} by J. Cavaillès, Fund.\ Math.\ \textbf{19}).
A comparative overview of the various definitions of finiteness is given by A. Tarski (\emph{Sur les ensembles finis}, Fund.\ Math.\ \textbf{6}\footnote{[\emph{Fundamenta Mathematicae \textbf{6} (1924) pp 45--95. Available from \url{https://www.impan.pl/en/publishing-house/journals-and-series/fundamenta-mathematicae/all/6/0/92568/sur-les-ensembles-finis} ---DMR}]}), whose own definition goes like this: A set is said to be finite if in every system of subsets at least one im System minimal is included.
The corresponding maximum condition is equivalent to this minimum condition through the transition to the complementary set; all properties of finite \textbf{[459]}
sets follow from both without using the selection postulate.
Tarski concludes that the above definition by Dedekind is equivalent to the minimal condition from the relation that also appears in a different version in Dedekind: $ab' \partof ab+[b']$.
In particular, Tarski gets from the above to the original Dedekind definition, while the reverse transition requires the axiom of choice.
Dedekind believed --- in the preface to the 2nd edition of ``Was sind und was sollen die Zahlen?'' --- that the proof of the agreement of the definitions required the full theory developed there.
How he thought about the transition in detail is shown in the following passage Letter to H. Weber:

\begin{quote}
``The shortest characterization of the finite and infinite is, as I believe, the one which I found on March 9, 1889 and in the preface (p. XI) to the second edition (1893) of the work, ``Was sind und was sollen die Zahlen?''.
I say it like this: `A system $S$ is called finite if there is a mapping of $S$ into itself through which no proper part of $S$ is mapped into itself; in the opposite case, $S$ is called an infinite system.'

But if one assumes that one already knows the natural number series and its laws completely, and one replaces the word ``called'' with the word ``is'' in the above, then this definition turns into a theorem that can be
proven like this:

Let $\phi$ be a mapping of a system $S$ into itself, through which no proper part of $S$ is mapped into itself.
I denote the image of an element $a$ or a part $A$ of $S$ with $a\phi$ or $A\phi$ (much more natural than $\phi(a)$ or $\phi(A)$).
If $a$ is any element of $S$, then all the images
\[
	a\phi, a\phi^2= (a\phi)\phi \ldots, a\phi^{n+1} = (a\phi^n)\phi \ldots
\]
are elements of $S$, so the system $A$ of all these images is also a part of $S$, and since $A\phi$ is the system of all images
\[
	(a\phi) \phi = a\phi^2, (a\phi^2)\phi = a\phi^3,
\]
is also a part of $A$, then $A$ is represented in itself by $\phi$; and consequently $A = S$.
Therefore a is also an element of $A$, so there is a smallest natural number $n$, that of the condition
\[
	a\phi^n = a
\]
suffices. Then $S$ is the system of $n$ elements
\[
	a\phi,a\phi^2,\ldots a\phi^n
\]
and these are different from each other.
For according to the definition of $n$, the last element is different from all previous ones; would also be $1 \leq r<s < n$ and
\[
	a\phi^r = a\phi^s
\]
that would be the case
\[
	(a\phi^r)\phi^{n-s} = (a\phi^s) \phi^{n-s}
\]
hence
\[
	a\phi^{r+n-s} = a\phi^n=a
\]
although $1<r+n-s<n$.
Finally, the fact that $S$ contains no elements other than these follows from $a\phi^{m + n} = a\phi^m$.
So $S$ is really a finite system (in the usual sense), and at the same time it follows that $\phi$ is a cyclic permutation of the $n$ elements of $S$, so it is also a similar (i.e. clearly reversible) mapping.

Conversely, a finite (in the usual sense) system $S$ consists of $n$ different elements
\[
	a_1, a_2 \ldots a_{n-1}, a_n
\]
and one defines a mapping $\phi$ of $S$ by
\[
a_n\phi = a_1, a_r\phi=a_{r+1}
\]
for $1\leq r<n$, then $S' = S$, so $\phi$ is a mapping from $S$ into itself, and it is easy to show that no proper part of $S$ is mapped into itself.
Because if a part $A$ of $S$ is represented by $\phi$ in itself and contains an element $a$, then $A$ must also contain all elements $a\phi, a\phi^2, a\phi^3, \ldots$, i.e. all elements of $S$, and therefore $= S$. Qed.''
\end{quote}

\begin{flushright}
\textbf{Noether.}\end{flushright}

\end{document}
